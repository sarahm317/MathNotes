\setchapterstyle{kao}
\setchapterpreamble[u]{\margintoc}
\chapter{Class Field Theory}
\labch{classFieldTheory}


\begin{definition}[Local field]
    \labdef{local-field}
    A local field is a field $K$ with a nontrivial absolute value $|\cdot|$ such that $K$ is locally compact with respect to $|\cdot|$.
\end{definition}

Requiring that $K$ is locally compact with respect to $|\cdot|$ implies that $K$ is complete with respect to $|\cdot|$. If $K = \R$ or $K = \C$, then $K$ is an archimedean local field. If the corresponding $|\cdot|$ has a discrete valuation with finite residue class field, then $K$ is a nonarchimedean local field.

\begin{definition}[Global field]
    \labdef{global-field}
    A global field is either:

    \begin{enumerate}[(1)]
        \item An algebraic number field.
        \item A function field in one variable over a finite field.
    \end{enumerate}
\end{definition}

A number field is always characteristic zero as it is defined as an extension over $\Q$. If the function field of an algebraic curve is taken over a finite field, it is the same as viewing it as a finite extension of some $\F_p(t)$ which is of nonzero (prime) characteristic.

\begin{proposition}
    A local field is the completion of some global field with respect to an absolute value.
\end{proposition}

Class field theory describes relationships between the abelian extensions of a number field $K$ and the structure of $\mathcal{O}_{K}$.

\begin{definition}[Unramified abelian extension]
    A maximal unramified abelian extension of $K$ is an extension $L$ that is unramified at all primes and every real embedding $K \hookrightarrow \R$ extends to a real embedding $L \hookrightarrow \R$.
\end{definition}

\marginnote{Not sure exactly what was being defined here... Should revist later.}

\begin{theorem}
    Let $L$ be a maximal unramified abelian extension of $K$. Then there exists a canonical isomorphism

    \[\textrm{Cl}(K)\xrightarrow{\cong} \textrm{Gal}(L/K).\]

\end{theorem}