\setchapterstyle{kao}
\setchapterpreamble[u]{\margintoc}
\chapter{Class Field Theory}
\labch{classFieldTheory}


\begin{definition}[Local field]
    \labdef{local-field}
    An archimedean local field is either $\R$ or $\C$. A nonarchimedean local field is a field complete with respect to a discrete valuation with the residue class field finite.
\end{definition}

\begin{definition}[Global field]
    \labdef{global-field}
    A global field is a number field or the function field of an algebraic curve over a finite field.
\end{definition}

A number field is always characteristic zero as it is defined as an extension over $\Q$. If the function field of an algebraic curve is taken over a finite field, it is the same as viewing it as a finite extension of some $\F_p(t)$ which is of nonzero (prime) characteristic.

\begin{proposition}
    A local field is the completion of some global field with respect to an absolute value.
\end{proposition}

Class field theory describes relationships between the abelian extensions of a number field $K$ and the structure of $\mathcal{O}_{K}$.