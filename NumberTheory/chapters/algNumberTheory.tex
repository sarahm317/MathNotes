\chapter{Algebraic Number Theory}

\section{Introduction}

This section is a brief review of algebraic number theory, particularly the concepts that are relevant to Class Field Theory.

\begin{definition}
\labdef{numberfield}
A number field $K$ is a finite field extension of $\mathbb{Q}$.
\end{definition}

\begin{definition}
    \labdef{alginteger}
    Let $K$ be a number field. An algebraic number $a \in K$ is called integral or an algebraic integer if $a$ is the root of some monic polynomial $f(x) = x^n + c_{n-1}x + \cdots + c_1x + c_0$.
\end{definition}

The set of algebraic integers over a number field $K$ is denoted by $\mathcal{O}_K$.

\begin{proposition}
    Let $K$ be a number field. Then, $\mathcal{O}_K$ is a ring and $K = \textrm{Frac}(\mathcal{O}_K)$.
\end{proposition}

\begin{proposition}
    The ring $\mathcal{O}_K$ is Noetherian, integrally closed, and every nonzero prime ideal is maximal.
\end{proposition}

\marginnote {
    Here integrally closed means that for any $a \in \textrm{Frac}(\mathcal{O}_K)$ that is integral over $\mathcal{O}_K$, it follows that $a \in \mathcal{O}_K$.
}

The above proposition here is equivalent to stating that $\mathcal{O}_K$ is a Dedekind domain.

\begin{theorem}[Unique Factorization of Ideals]
    Every nonzero ideal $\mathfrak{a} \not\subseteq \mathcal{O}_K$ can be uniquely written in the form
        \[\mathfrak{a} = \mathfrak{p}_1^{r_1}\cdots\mathfrak{p}_m^{r_m}\]
    where $m \geq 1$, each $r_i \in \mathbb{N}$, and  $\mathfrak{p}_1, \ldots, \mathfrak{p}_m$ are distinct, nonzero prime ideals of $\mathcal{O}_K$.
\end{theorem}