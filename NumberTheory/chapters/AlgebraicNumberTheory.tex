\setchapterstyle{kao}
\setchapterpreamble[u]{\margintoc}
\chapter{Algebraic Number Theory}
\labch{algNumberTheory}

\section{General Definitions}

\begin{definition}[Number field]
    \labdef{numberField}
    A number field is a finite field extension over $\Q$.
\end{definition}

\begin{definition}[Algebraic integer]
    \labdef{algebraicInteger}
    Let $K$ be a number field. An algebraic number $a \in K$ is called integral or an algebraic integer of $K$ if $f(a) = 0$ for some monic polynomial $f$ with coefficients in $\Z$. Denote the set of algebraic integers in $K$ by $\mathcal{O}_{K}$.
\end{definition}

\marginnote{
    The notes here about algebraic number theory are very brief -- the recommended texts for a more in depth reading are:
    \begin{itemize}
        \item Algebraic Number Theory Chapters I, II (Neukirch)
        \item Algebraic Number Theory Notes (Milne)
    \end{itemize}
}

\begin{proposition}
    Let $K$ be a number field. Then $\mathcal{O}_{K}$ is a ring and $K = \textrm{Frac}(\mathcal{O}_{K})$.
\end{proposition}

\begin{proposition}
    The ring $\mathcal{O}_{K}$ is Noetherian, integrally closed, and every nonzero prime ideal of $\mathcal{O}_{K}$ is maximal.
\end{proposition}

Notice that the results presented in the proposition above imply that $\mathcal{O}_{K}$ is a Dedekind domain, using one of the many equivalent defintions of a Dedekind domain.

\begin{theorem}[Unique Factorization of Ideals]
    \labthm{uniqueFactIdeals}
    Every nonzero ideal $\mathfrak{a} \not\sq \mathcal{O}_{K}$ can be uniquely written as
        \[\mathfrak{a} = \mathfrak{p}_1^{r_1}\cdots \mathfrak{p}_m^{r_m}\]

    where $m \geq 1$, $\mathfrak{p}_1,\ldots, \mathfrak{p}_m$ are distinct nonzero prime ideals of $\mathcal{O}_{K}$, and $r_1, \ldots, r_m \in \N$.
\end{theorem}

\marginnote{
    \refthm{uniqueFactIdeals} is actually true for any Dedekind domain, but we just focus on this specific case here.
}

\begin{definition}[Trace, Norm]
    \labdef{trace}
    \labdef{norm}
    Suppose that $\Q \sq K \sq L$ is an extension of fields. Let $a \in L$ and view $L$ as a $K$-vector space to consider the linear transformation
        \[T_a: L \to L\]
        \[x\mapsto ax. \]
    Define the trace and norm for $a$ as
        \[\textrm{Tr}_{L/K}(a) = \textrm{Tr}(T_a) \in K\]
    and
        \[\textrm{Nm}_{L/K}(a) = \textrm{det}(T_a) \in K.\]
\end{definition}

With trace and norm defined as in \refdef{trace}, we obtain a bi-$K$-linear pairing:

    \[\langle \cdot ,\cdot \rangle_{L/K}: L \times K \to K\]
given by
    \[\langle a, b\rangle_{L/K} = \textrm{Tr}_{L/K}(ab).\]
\begin{definition}
    \labdef{discriminant}
    Let $\alpha_1, \ldots, \alpha_n$ be a basis of $L$ over $K$. The discriminant of $\alpha_1, \ldots, \alpha_n$ is defined as
        \[D(\alpha_1, \ldots, \alpha_n) = \det\left(\begin{pmatrix}
            \langle \alpha_i, \alpha_j\rangle
        \end{pmatrix}_{1\leq i,j\leq n}\right).\]
    The discriminant of $L/K$ is denoted by $D_{L/K}$ and is the ideal of $\mathcal{O}_{K}$ generated by
        \[\{D(\alpha_1,\ldots, \alpha_n): \alpha_1, \ldots, \alpha_n \textrm{is a basis of $L/K$ contained in $\mathcal{O}_{L}$}\}.\]
\end{definition}

\marginnote{
    The matrix \[\begin{pmatrix}
        \langle \alpha_i, \alpha_j\rangle
    \end{pmatrix}_{1\leq i,j\leq n}\]
    is an $n\times n$ matrix, with entries in $K$.
}

For $K/\Q$, $\mathcal{O}_{\Q} = \Z$ and therefore is a PID. So, $\mathcal{O}_{K}$ is a free $\Z$-module of rank $n = [K:\Q]$. For any $\Z$-basis $\alpha_1, \ldots, \alpha_n$ of $\mathcal{O}_{K}$,
    \[D_{K/\Q} = (D(\alpha_1, \ldots, \alpha_n)).\]

\begin{definition}[Ramification index, Residue class degree/Intertia degree]
    \labdef{ramification-index}
    \labdef{intertia-degree}
    \labdef{residue-class-degree}
    Let $L/K$ be an extension of number fields, $\wp  \sq \mathcal{O}_{L}$ a nonzero prime ideal, and define $\mathfrak{p} = \wp \cap \mathcal{O}_{K} \sq \mathcal{O}_{K}$. Write the prime factorization of $\mathfrak{p}\mathcal{O}_{L}$ as
    \[\mathfrak{p}\mathcal{O}_{L} = \wp _1^{e_1}\cdots \wp _m^{e_m}\]
    where $\wp _1 = \wp $. The ramification index of $\wp $ over $\mathfrak{p}$, denoted by $e(\wp /\mathfrak{p})$, is defined to be $e_1$ (as given in the prime factorization). The residue class degree, or the intertia degree, of $\wp $ of $\mathfrak{p}$, denoted by $e(\wp /\mathfrak{p})$, is defined to be $[\mathcal{O}_{L}/\wp : \mathcal{O}_{K}/\mathfrak{p}]$.
\end{definition}

\begin{definition}[Ramified]
    \labdef{ramified}
    Let $L/K$ be an extension of number fields and $\mathfrak{p} \sq \mathcal{O}_{K}$ a nonzero prime ideal. We say $\mathfrak{p}$ is ramified in $L$ or $L/K$ is ramified at $\mathfrak{p}$ if $e(\wp /\mathfrak{p}) > 1$ for some $\wp  \sq \mathcal{O}_{L}$ satisfying $\mathfrak{p} = \wp \cap \mathcal{O}_{K}$. We say $\mathfrak{p}$ is unramified in $L$ or $L/K$ is unramified at $\mathfrak{p}$ if $e(\wp /\mathfrak{p}) =1$ for every $\wp  \sq \mathcal{O}_{L}$ where $\mathfrak{p} = \wp \cap \mathcal{O}_{K}$.
\end{definition}

\begin{definition}[Splits, Splits completely]
    \labdef{splits}
    Let $L/K$ be an extension of number fields and $\mathfrak{p} \sq \mathcal{O}_{K}$ a nonzero prime ideal. We say $\mathfrak{p}$ splits or splits completely in $L$ if $e(\wp /\mathfrak{p}) = f(\wp /\mathfrak{p}) = 1$ for every $\wp  \sq \mathcal{O}_{L}$ with $\wp  \cap \mathcal{O}_{K} = \mathfrak{p}$.
\end{definition}

\begin{definition}[Inert]
    \labdef{inert}
    Let $L/K$ be an extension of number fields and $\mathfrak{p} \sq \mathcal{O}_{K}$ a nonzero prime ideal. We say that $\mathfrak{p}$ is inert in $L$ if $\mathfrak{p}\mathcal{O}_{L}$ is a prime ideal of $\mathcal{O}_{L}$.
\end{definition}

From these definitions, one can derive the following identity: if $\mathfrak{p} \mathcal{O}_{L} = \wp _1^{e_1}\cdots \wp _m^{e_m}$ then 

    \[[L:K] = \sum_{j=1}^m e(\wp _j/\mathfrak{p}_j)f(\mathit{p}_j/\mathfrak{p}_j).\]

\begin{theorem}
    \labthm{unramified-criterion}
    The extension $L/K$ is unramified at $\mathfrak{p} \sq \mathcal{O}_{K}$ if and only if $\mathfrak{p}$ does not divide $D_{L/K}$.  That is, $D_{L/K} \not\sq \mathfrak{p}$ if and only if $\mathfrak{p}$ and $D_{L/K}$ are coprime ($\mathfrak{p} + D_{L/K} = \mathcal{O}_{K}$).
\end{theorem}

\begin{theorem}[Minkowski]
    \labthm{Minkowski}
    $\Q$ has non nontrivial extension that is unramified at all primes. Equivalently, every $D_{K/\Q} \neq \pm 1$.
\end{theorem}

Note that \refthm{Minkowski} is not true for a general number field $K$:

\begin{example}
    Let $K = \Q(\sqrt{-5})$ and $L = K(\sqrt{-1})$ so that $L/K$ is an extension of number fields. Then, $\mathcal{O}_{K} = \Z[\sqrt{-5}]$ and $L = K(\sqrt{5})$. To see that $L/K$ is unramified at all primes, we apply \refthm{unramified-criterion} and show that $D_{L/K} = \mathcal{O}_{K}$.\\

    The remainder of this example is just some computations regarding the discriminant and two diferent $K$-bases of $L$.
\end{example}

\begin{definition}[Fractional ideal]
    \labdef{fractionalIdeal}
    A fractional ideal of $K$ is a nonzero finitely generated $\mathcal{O}_{K}$-submodule of $K$.
\end{definition}

One can define a multiplication on the collection of fractional ideals of $K$: if $\mathfrak{a}_1, \ldots, \mathfrak{a}_n$ are all fractional ideals of $K$, then the product is the $\mathcal{O}_{K}$-submodule of $K$ generated by $\{a_1\cdots a_n | a_j \in \mathfrak{a}_j\}$.

\begin{proposition}
    The collection of fractional ideals of $K$ forms an abelian group under the multiplication of fractional ideals.
    With this structure, the identity is $\mathcal{O}_{K}$ and the inverse of $\mathfrak{a}$ is $\mathfrak{a}^{-1} = \{x \in K| x \mathfrak{a} \sq \mathcal{O}_{K}\}$.
\end{proposition}



%%%% There should be some notes here -- think i need to pull from desktop

\begin{proposition}
    Let $K$ be a number field. Every fractional ideal $\mathfrak{a}$ of $K$ can be written uniquely in the form 
        \[\mathfrak{a} = \prod_{\mathfrak{p}} \mathfrak{p}^{r_\mathfrak{p}}\]
    where the product is taken over all the nonzero prime ideals of $\mathcal{O}_{K}$, each $r_\mathfrak{p} \in \Z$, and almost every $r_\mathfrak{p}$ is zero.
\end{proposition}

\begin{remark}
    With these definitions, $I_K$ is the free abelian group on the set of nonzero prime ideals of $\mathcal{O}_{K}$.
\end{remark}

Define a subgroup of $I_K$ by
    \[P_K = \left\{ (a) = a \mathcal{O}_{K}: a \in K^\times \right\}.\]

\begin{definition}[Ideal class group, Class group]
    \labdef{class-group, ideal-class-group}
    The ideal class group or class group of $K$ is defined as 
        \[\textrm{Cl}(K) = I_K/P_K.\]
\end{definition}

\begin{theorem}
    For any number field $K$, the class group $\textrm{Cl}(K)$ is finite.
\end{theorem}

\begin{definition}[Class number]
    \labdef{class-number}
    The class number of a number field $K$ is the order of the class group $\textrm{Cl}(K)$.
\end{definition}

The proof that the class number of a given number field is indeed finite uses Minkowski Theory.

For a number field $K$, let $r_k$ denote the number of real embeddings of $K$ into $\R$ and $s_k$ denote the number of pairs of complex embeddings of $K$ into $\C$. Here we are assuming that $s_k$ is counting the pairs of embeddings that are not strictly contained in $\R$. Note that the complex embeddings occur in pairs through complex conjugation.

\begin{theorem}[Dirichlet's Unit Theorem]
    Suppose that $K$ is a number field and $\mu(K)$ is the finite group of roots of unity that are contained in $K$. Then,
        \[\mathcal{O}_{K}^\times \cong \Z^{r_k + s_k -1} \times \mu(K).\]
\end{theorem}

\begin{definition}[Decomposition group]
    \labdef{decomposition-group}
    Suppose that $L/K$ is a Galois extension of number fields, $\wp \sq L$ is a prime ideal, and $\mathfrak{p} = \wp \cap \mathcal{O}_{K}$. The decomposition group of $\wp$ is the set 
        \[G_\wp = \{\sigma \in \textrm{Gal}(L/K): \sigma(\wp) = \wp\}.\]
\end{definition}

\marginnote{Need to check the assumptions here -- where is $\wp$ living? Nonzero?}

\begin{definition}[Inertia group]
    \labdef{intertia-group}
    Let $\kappa = \mathcal{O}_{K}/\mathfrak{p}$ and $\lambda = \mathcal{O}_{L}/\wp$. The kernel of the map
        \[G_\wp \to \textrm{Aut}(\lambda/\kappa)\]
    is the inertia group of $\wp$ and is denoted by $I_\wp$. 
\end{definition}

\section{Valuations and Absolute Values}

In general, assume hereafter that $p$ denotes some prime number.

\begin{definition}[$p$-adic absolute value, $p$-adic norm]
    \labdef{padic-norm}
    The $p$-adic absolute value or norm of $\Q$
        \[|\cdot |_p: \Q \to \R\]
    is defined by
        \[\left| p^m \frac{a}{b}\right|_{p} = p^{-m}\]
    where both $a$ and $b$ are coprime to $p$. Set $|0|_p = 0$.
\end{definition}

\begin{proposition}
    The $p$-adic norm is indeed a norm. That is:

    \begin{enumerate}
        \item $|a|_p > 0$ for all $a \in \Q^\times$
        \item $|ab|_p = |a|_p|b|_p$
        \item $|a + b|_p \leq |a|_p + |b|_p$
    \end{enumerate}
\end{proposition}

The $p$-adic norm actually satsifies a stronger version of the triangle inequality:
$|a + b|_p \leq \max\{ |a|_p ,|b|_p\}$. Since we have now equipped $\Q$ with a norm, it can be viewed as a topological space and thus there is a notion of convergence and Cauchy sequences. In particular, we are interested in studying the completion of $\Q$ with respect to a given $p$-adic norm.

\begin{definition}[$p$-adic numbers]
    Let $\Q_p$ be the completion of $\Q$ with respect to the $p$-adic norm. The elements of $\Q_p$ are called the $p$-adic numbers.
\end{definition}

Using properties of limits and the fact that every element of $\Q_p$ can be represented as the limit of a sequence of points in $\Q$, the addition and multiplication of $\Q$ can be naturally extended to $\Q_p$. Likewise, the norm $|\cdot|_p$ can be extended to a norm on $\Q_p$. With these operations, $\Q_p$ is a field that contains $\Q$ as a subfield.