\setchapterstyle{kao}
\setchapterpreamble[u]{\margintoc}
\chapter{Algebraic Number Theory}
\labch{algNumberTheory}

\begin{definition}
    \labdef{numberField}
    A number field is a finite field extension over $\Q$.
\end{definition}

\begin{definition}
    \labdef{algebraicInteger}
    Let $K$ be a number field. An algebraic number $a \in K$ is called integral or an algebraic integer of $K$ if $f(a) = 0$ for some monic polynomial $f$ with coefficients in $\Z$. Denote the set of algebraic integers in $K$ by $\mathcal{O}_{K}$.
\end{definition}

\marginnote{
    The notes here about algebraic number theory are very brief -- the recommended texts for a more in depth reading are:
    \begin{itemize}
        \item Algebraic Number Theory Chapters I, II (Neukirch)
        \item Algebraic Number Theory Notes (Milne)
    \end{itemize}
}

\begin{proposition}
    Let $K$ be a number field. Then $\mathcal{O}_{K}$ is a ring and $K = \textrm{Frac}(\mathcal{O}_{K})$.
\end{proposition}

\begin{proposition}
    The ring $\mathcal{O}_{K}$ is Noetherian, integrally closed, and every nonzero prime ideal of $\mathcal{O}_{K}$ is maximal.
\end{proposition}

Notice that the results presented in the proposition above imply that $\mathcal{O}_{K}$ is a Dedekind domain, using one of the many equivalent defintions of a Dedekind domain.

\begin{theorem}[Unique Factorization of Ideals]
    \labthm{uniqueFactIdeals}
    Every nonzero ideal $\mathfrak{a} \not\sq \mathcal{O}_{K}$ can be uniquely written as
        \[\mathfrak{a} = \mathfrak{p}_1^{r_1}\cdots \mathfrak{p}_m^{r_m}\]

    where $m \geq 1$, $\mathfrak{p}_1,\ldots, \mathfrak{p}_m$ are distinct nonzero prime ideals of $\mathcal{O}_{K}$, and $r_1, \ldots, r_m \in \N$.
\end{theorem}

\marginnote{
    \refthm{uniqueFactIdeals} is actually true for any Dedekind domain, but we just focus on this specific case here.
}

\begin{definition}
    \labdef{traceNorm}
    Suppose that $\Q \sq K \sq L$ is an extension of fields. Let $a \in L$ and view $L$ as a $K$-vector space to consider the linear transformation
        \[T_a: L \to \]
        \[x\mapsto ax. \]
    Define the trace and norm for $a$ as
        \[\textrm{Tr}_{L/K}(a) = \textrm{Tr}(T_a) \in K\]
    and
        \[\textrm{Nm}_{L/K}(a) = \textrm{det}(T_a) \in K.\]
\end{definition}

With trace and norm defined as in \refdef{traceNorm}, we obtain a bi-$K$-linear pairing:

    \[\langle , \rangle_{L/K}: L \times K \to K\]
given by
    \[\langle a, \rangle_{L/K} = \textrm{Tr}_{L/K}(ab).\]
\begin{definition}
    Let $\alpha_1, \ldots, \alpha_n$ be a basis of $L$ over $K$. The discriminant of $\alpha_1, \ldots, \alpha_n$ is defined as
        \[D(\alpha_1, \ldots, \alpha_n) = \det\left(\begin{pmatrix}
            \langle \alpha_i, \alpha_j\rangle
        \end{pmatrix}_{1\leq i,j\leq n}\right).\]
    The discriminant of $L/K$ is denoted by $D_{L/K}$ and is the ideal of $\mathcal{O}_{K}$ generated by
        \[\{D(\alpha_1,\ldots, \alpha_n): \alpha_1, \ldots, \alpha_n \textrm{is a basis of $L/K$ contained in $\mathcal{O}_{L}$}\}.\]
\end{definition}

\marginnote{
    The matrix \[\begin{pmatrix}
        \langle \alpha_i, \alpha_j\rangle
    \end{pmatrix}_{1\leq i,j\leq n}\]
    is an $n\times n$ matrix, with entries in $K$.
}

For $K/\Q$, $\mathcal{O}_{\Q} = \Z$ and therefore is a PID. So, $\mathcal{O}_{K}$ is a free $\Z$-module of rank $n = [K:\Q]$. For any $\Z$-basis $\alpha_1, \ldots, \alpha_n$ of $\mathcal{O}_{K}$,
    \[D_{K/\Q} = (D(\alpha_1, \ldots, \alpha_n)).\]