\setchapterstyle{kao}
\setchapterpreamble[u]{\margintoc}
\chapter{Algebraic Number Theory}
\labch{algNumberTheory}

\begin{definition}[Number field]
    \labdef{numberField}
    A number field is a finite field extension over $\Q$.
\end{definition}

\begin{definition}[Algebraic integer]
    \labdef{algebraicInteger}
    Let $K$ be a number field. An algebraic number $a \in K$ is called integral or an algebraic integer of $K$ if $f(a) = 0$ for some monic polynomial $f$ with coefficients in $\Z$. Denote the set of algebraic integers in $K$ by $\mathcal{O}_{K}$.
\end{definition}

\marginnote{
    The notes here about algebraic number theory are very brief -- the recommended texts for a more in depth reading are:
    \begin{itemize}
        \item Algebraic Number Theory Chapters I, II (Neukirch)
        \item Algebraic Number Theory Notes (Milne)
    \end{itemize}
}

\begin{proposition}
    Let $K$ be a number field. Then $\mathcal{O}_{K}$ is a ring and $K = \textrm{Frac}(\mathcal{O}_{K})$.
\end{proposition}

\begin{proposition}
    The ring $\mathcal{O}_{K}$ is Noetherian, integrally closed, and every nonzero prime ideal of $\mathcal{O}_{K}$ is maximal.
\end{proposition}

Notice that the results presented in the proposition above imply that $\mathcal{O}_{K}$ is a Dedekind domain, using one of the many equivalent defintions of a Dedekind domain.

\begin{theorem}[Unique Factorization of Ideals]
    \labthm{uniqueFactIdeals}
    Every nonzero ideal $\mathfrak{a} \not\sq \mathcal{O}_{K}$ can be uniquely written as
        \[\mathfrak{a} = \mathfrak{p}_1^{r_1}\cdots \mathfrak{p}_m^{r_m}\]

    where $m \geq 1$, $\mathfrak{p}_1,\ldots, \mathfrak{p}_m$ are distinct nonzero prime ideals of $\mathcal{O}_{K}$, and $r_1, \ldots, r_m \in \N$.
\end{theorem}

\marginnote{
    \refthm{uniqueFactIdeals} is actually true for any Dedekind domain, but we just focus on this specific case here.
}

\begin{definition}[Trace, Norm]
    \labdef{trace}
    \labdef{norm}
    Suppose that $\Q \sq K \sq L$ is an extension of fields. Let $a \in L$ and view $L$ as a $K$-vector space to consider the linear transformation
        \[T_a: L \to L\]
        \[x\mapsto ax. \]
    Define the trace and norm for $a$ as
        \[\textrm{Tr}_{L/K}(a) = \textrm{Tr}(T_a) \in K\]
    and
        \[\textrm{Nm}_{L/K}(a) = \textrm{det}(T_a) \in K.\]
\end{definition}

With trace and norm defined as in \refdef{trace}, we obtain a bi-$K$-linear pairing:

    \[\langle \cdot ,\cdot \rangle_{L/K}: L \times K \to K\]
given by
    \[\langle a, b\rangle_{L/K} = \textrm{Tr}_{L/K}(ab).\]
\begin{definition}
    \labdef{discriminant}
    Let $\alpha_1, \ldots, \alpha_n$ be a basis of $L$ over $K$. The discriminant of $\alpha_1, \ldots, \alpha_n$ is defined as
        \[D(\alpha_1, \ldots, \alpha_n) = \det\left(\begin{pmatrix}
            \langle \alpha_i, \alpha_j\rangle
        \end{pmatrix}_{1\leq i,j\leq n}\right).\]
    The discriminant of $L/K$ is denoted by $D_{L/K}$ and is the ideal of $\mathcal{O}_{K}$ generated by
        \[\{D(\alpha_1,\ldots, \alpha_n): \alpha_1, \ldots, \alpha_n \textrm{is a basis of $L/K$ contained in $\mathcal{O}_{L}$}\}.\]
\end{definition}

\marginnote{
    The matrix \[\begin{pmatrix}
        \langle \alpha_i, \alpha_j\rangle
    \end{pmatrix}_{1\leq i,j\leq n}\]
    is an $n\times n$ matrix, with entries in $K$.
}

For $K/\Q$, $\mathcal{O}_{\Q} = \Z$ and therefore is a PID. So, $\mathcal{O}_{K}$ is a free $\Z$-module of rank $n = [K:\Q]$. For any $\Z$-basis $\alpha_1, \ldots, \alpha_n$ of $\mathcal{O}_{K}$,
    \[D_{K/\Q} = (D(\alpha_1, \ldots, \alpha_n)).\]

\begin{definition}
    \labdef{ramification-index}
    Let $L/K$ be an extension of number fields, $\mathcal{p} \sq \mathcal{O}_{L}$ a nonzero prime ideal, and define $\mathfrak{p} = \mathcal{p}\cap \mathcal{O}_{K} \sq \mathcal{O}_{K}$. Write the prime factorization of $\mathfrak{p}\mathcal{O}_{L}$ as
    \[\mathfrak{p}\mathcal{O}_{L} = \mathcal{p}_1^{e_1}\cdots \mathcal{p}_m^{e_m}\]
    where $\mathcal{p}_1 = \mathcal{p}$. The ramification index of $\mathcal{p}$ over $\mathfrak{p}$, denoted by $e(\mathcal{p}/\mathfrak{p})$, is defined to be $e_1$ (as given in the prime factorization). The residue class degree, or the intertia degree, of $\mathcal{p}$ of $\mathfrak{p}$, denoted by $e(\mathcal{p}/\mathfrak{p})$, is defined to be $[\mathcal{O}_{L}/\mathcal{p}: \mathcal{O}_{K}/\mathfrak{p}]$.
\end{definition}

\begin{definition}
    \labdef{ramified}
    Let $L/K$ be an extension of number fields and $\mathfrak{p} \sq \mathcal{O}_{K}$ a nonzero prime ideal. We say $\mathfrak{p}$ is ramified in $L$ or $L/K$ is ramified at $\mathfrak{p}$ if $e(\mathcal{p}/\mathfrak{p}) > 1$ for some $\mathcal{p} \sq \mathcal{O}_{L}$ satisfying $\mathfrak{p} = \mathcal{p}\cap \mathcal{O}_{K}$. We say $\mathfrak{p}$ is unramified in $L$ or $L/K$ is unramified at $\mathfrak{p}$ if $e(\mathcal{p}/\mathfrak{p}) =1$ for every $\mathcal{p} \sq \mathcal{O}_{L}$ where $\mathfrak{p} = \mathcal{p}\cap \mathcal{O}_{K}$.
\end{definition}

\begin{definition}
    \labdef{splits}
    Let $L/K$ be an extension of number fields and $\mathfrak{p} \sq \mathcal{O}_{K}$ a nonzero prime ideal. We say $\mathfrak{p}$ splits or splits completely in $L$ if $e(\mathcal{p}/\mathfrak{p}) = f(\mathcal{p}/\mathfrak{p}) = 1$ for every $\mathcal{p} \sq \mathcal{O}_{L}$ with $\mathcal{p} \cap \mathcal{O}_{K} = \mathfrak{p}$.
\end{definition}

\begin{definition}
    \labdef{inert}
    Let $L/K$ be an extension of number fields and $\mathfrak{p} \sq \mathcal{O}_{K}$ a nonzero prime ideal. We say that $\mathfrak{p}$ is inert in $L$ if $\mathfrak{p}\mathcal{O}_{L}$ is a prime ideal of $\mathcal{O}_{L}$.
\end{definition}

From these definitions, one can derive the following identity: if $\mathfrak{p} \mathcal{O}_{L} = \mathcal{p}_1^{e_1}\cdots \mathcal{p}_m^{e_m}$ then 

    \[[L:K] = \sum_{j=1}^m e(\mathcal{p}_j/\mathfrak{p}_j)f(\mathit{p}_j/\mathfrak{p}_j).\]

%%%% There should be some notes here -- think i need to pull from desktop